\documentclass[12pt]{report}
\usepackage[utf8x]{inputenc}

%Warning: tipa declares many non-standard macros used by utf8x to
%interpret utf8 characters but extra packages might have to be added
%such as "textgreek" for Greek letters not already in tipa
%or "stmaryrd" for mathematical symbols.
%Utf8 codes missing a LaTeX interpretation can be defined by using
%\DeclareUnicodeCharacter{code}{interpretation}.
%Use coqdoc's option -p to add new packages or declarations.
\usepackage{tipa}

\usepackage[T1]{fontenc}
\usepackage{fullpage}
\usepackage{coqdoc}
\usepackage{amsmath,amssymb}
\usepackage{url}
\begin{document}
%%%%%%%%%%%%%%%%%%%%%%%%%%%%%%%%%%%%%%%%%%%%%%%%%%%%%%%%%%%%%%%%%
%% This file has been automatically generated with the command
%% coqdoc --latex --utf8 Lists.v 
%%%%%%%%%%%%%%%%%%%%%%%%%%%%%%%%%%%%%%%%%%%%%%%%%%%%%%%%%%%%%%%%%
\coqlibrary{Lists}{Library }{Lists}

\begin{coqdoccode}
\end{coqdoccode}
\section{Lists: 使用结构化的数据}





  本节介绍\textit{'列表'(List)\_数据类型。
  下一节介绍\textit{'函数式程序设计' (Functional Programming; FP)\_范型。


  为什么要先介绍列表呢?
  列表是函数式程序设计中的基础数据类型。
  最早的(?)的函数式程序设计语言 Lisp 的含义即是 ``LISt Processor''。
\begin{coqdoccode}
\coqdocemptyline
\coqdocnoindent
\coqdocvar{From} \coqdocvar{LF} \coqdockw{Require} \coqdockw{Export} \coqdockw{Induction}.\coqdoceol
\coqdocnoindent
\coqdockw{Module} \coqdocvar{NatList}.\coqdoceol
\end{coqdoccode}
\section{自然数序对}





  在定义列表数据类型之前,我们先热热身,
  定义简单的自然数 \_序对 (Ordered Pair)\_。
  它只有一种构造方式,即将构造函数 \coqdocvar{pair} 作用到两个自然数 \coqdocvar{n1} \coqdocvar{n2} 上。 
\begin{coqdoccode}
\coqdocemptyline
\coqdocnoindent
\coqdockw{Inductive} \coqdocvar{natprod} : \coqdockw{Type} :=\coqdoceol
\coqdocindent{1.00em}
\ensuremath{|} \coqdocvar{pair} (\coqdocvar{n1} \coqdocvar{n2} : \coqdocvar{nat}).\coqdoceol
\coqdocemptyline
\end{coqdoccode}
  注意: 我们将该类型命名为 natprod,
  其中 prod 表示 \textit{'乘积' (Product)\_ 类型。
\begin{coqdoccode}
\coqdocnoindent
\coqdockw{Check} (\coqdocvar{pair} 3 5).\coqdoceol
\coqdocemptyline
\end{coqdoccode}
  函数 \coqdocvar{fst} 与 \coqdocvar{snd} 分别用于提取有序对的第一个和第二个分量。
\begin{coqdoccode}
\coqdocemptyline
\coqdocnoindent
\coqdockw{Definition} \coqdocvar{fst} (\coqdocvar{p} : \coqdocvar{natprod}) : \coqdocvar{nat} :=\coqdoceol
\coqdocindent{1.00em}
\coqdockw{match} \coqdocvar{p} \coqdockw{with}\coqdoceol
\coqdocindent{1.00em}
\ensuremath{|} \coqdocvar{pair} \coqdocvar{x} \coqdocvar{y} \ensuremath{\Rightarrow} \coqdocvar{x}\coqdoceol
\coqdocindent{1.00em}
\coqdockw{end}.\coqdoceol
\coqdocemptyline
\coqdocnoindent
\coqdockw{Definition} \coqdocvar{snd} (\coqdocvar{p} : \coqdocvar{natprod}) : \coqdocvar{nat} :=\coqdoceol
\coqdocindent{1.00em}
\coqdockw{match} \coqdocvar{p} \coqdockw{with}\coqdoceol
\coqdocindent{1.00em}
\ensuremath{|} \coqdocvar{pair} \coqdocvar{x} \coqdocvar{y} \ensuremath{\Rightarrow} \coqdocvar{y}\coqdoceol
\coqdocindent{1.00em}
\coqdockw{end}.\coqdoceol
\coqdocemptyline
\coqdocnoindent
\coqdockw{Compute} (\coqdocvar{fst} (\coqdocvar{pair} 3 5)).\coqdoceol
\coqdocemptyline
\end{coqdoccode}
  在数学上,我们使用 (\coqdocvar{x},\coqdocvar{y}) 表示有序对 \coqdocvar{pair} \coqdocvar{x} \coqdocvar{y}。
\begin{coqdoccode}
\coqdocemptyline
\coqdocnoindent
\coqdockw{Notation} "( x , y )" := (\coqdocvar{pair} \coqdocvar{x} \coqdocvar{y}).\coqdoceol
\coqdocemptyline
\coqdocnoindent
\coqdockw{Compute} (\coqdocvar{fst} (3,5)).\coqdoceol
\coqdocemptyline
\coqdocnoindent
\coqdockw{Definition} \coqdocvar{fst'} (\coqdocvar{p} : \coqdocvar{natprod}) : \coqdocvar{nat} :=\coqdoceol
\coqdocindent{1.00em}
\coqdockw{match} \coqdocvar{p} \coqdockw{with}\coqdoceol
\coqdocindent{1.00em}
\ensuremath{|} (\coqdocvar{x},\coqdocvar{y}) \ensuremath{\Rightarrow} \coqdocvar{x} \coqdoceol
\coqdocindent{1.00em}
\coqdockw{end}.\coqdoceol
\coqdocemptyline
\coqdocnoindent
\coqdockw{Definition} \coqdocvar{snd'} (\coqdocvar{p} : \coqdocvar{natprod}) : \coqdocvar{nat} :=\coqdoceol
\coqdocindent{1.00em}
\coqdockw{match} \coqdocvar{p} \coqdockw{with}\coqdoceol
\coqdocindent{1.00em}
\ensuremath{|} (\coqdocvar{x},\coqdocvar{y}) \ensuremath{\Rightarrow} \coqdocvar{y}\coqdoceol
\coqdocindent{1.00em}
\coqdockw{end}.\coqdoceol
\coqdocemptyline
\coqdocnoindent
\coqdockw{Definition} \coqdocvar{swap\_pair} (\coqdocvar{p} : \coqdocvar{natprod}) : \coqdocvar{natprod} :=\coqdoceol
\coqdocindent{1.00em}
\coqdockw{match} \coqdocvar{p} \coqdockw{with}\coqdoceol
\coqdocindent{1.00em}
\ensuremath{|} (\coqdocvar{x},\coqdocvar{y}) \ensuremath{\Rightarrow} (\coqdocvar{y},\coqdocvar{x})\coqdoceol
\coqdocindent{1.00em}
\coqdockw{end}.\coqdoceol
\coqdocemptyline
\end{coqdoccode}
  由于 \coqdocvar{natprod} 也是归纳类型 (使用 Inductive 定义),
  因此我们可以使用 \coqdoctac{destruct} 对 \coqdocvar{natprod} 类型的值分情形讨论。
  又由于 \coqdocvar{natprod} 只有一个构造函数 \coqdocvar{pair},
  因此使用 \coqdoctac{destruct} 时只会产生一个子目标。
  另外,\coqdocvar{pair} 有两个参数,
  所以可以使用 \coqdoctac{destruct} 的 \coqdockw{as} [\coqdocvar{f} \coqdocvar{s}] 子句
  匹配并记录有序对的两个分量。 
\begin{coqdoccode}
\coqdocnoindent
\coqdockw{Theorem} \coqdocvar{surjective\_pairing} : \coqdockw{\ensuremath{\forall}} (\coqdocvar{p} : \coqdocvar{natprod}),\coqdoceol
\coqdocindent{1.00em}
\coqdocvar{p} = (\coqdocvar{fst} \coqdocvar{p}, \coqdocvar{snd} \coqdocvar{p}).\coqdoceol
\coqdocnoindent
\coqdockw{Proof}.\coqdoceol
\coqdocindent{1.00em}
\coqdoctac{intros} \coqdocvar{p}.\coqdoceol
\coqdocindent{1.00em}
\coqdoctac{destruct} \coqdocvar{p} \coqdockw{as} [\coqdocvar{n} \coqdocvar{m}]. \coqdocindent{1.00em}
\coqdoctac{simpl}. \coqdoctac{reflexivity}.\coqdoceol
\coqdocnoindent
\coqdockw{Qed}.\coqdoceol
\coqdocemptyline
\end{coqdoccode}
\paragraph{练习:1 星, standard (snd\_fst\_is\_swap)}

\begin{coqdoccode}
\coqdocnoindent
\coqdockw{Theorem} \coqdocvar{snd\_fst\_is\_swap} : \coqdockw{\ensuremath{\forall}} (\coqdocvar{p} : \coqdocvar{natprod}),\coqdoceol
\coqdocindent{1.00em}
(\coqdocvar{snd} \coqdocvar{p}, \coqdocvar{fst} \coqdocvar{p}) = \coqdocvar{swap\_pair} \coqdocvar{p}.\coqdoceol
\coqdocnoindent
\coqdockw{Proof}.\coqdoceol
\coqdocnoindent
\coqdocvar{Admitted}.\coqdoceol
\end{coqdoccode}
\ensuremath{\Box} 

\paragraph{练习:1 星, standard, optional (fst\_swap\_is\_snd)}

\begin{coqdoccode}
\coqdocnoindent
\coqdockw{Theorem} \coqdocvar{fst\_swap\_is\_snd} : \coqdockw{\ensuremath{\forall}} (\coqdocvar{p} : \coqdocvar{natprod}),\coqdoceol
\coqdocindent{1.00em}
\coqdocvar{fst} (\coqdocvar{swap\_pair} \coqdocvar{p}) = \coqdocvar{snd} \coqdocvar{p}.\coqdoceol
\coqdocnoindent
\coqdockw{Proof}.\coqdoceol
\coqdocnoindent
\coqdocvar{Admitted}.\coqdoceol
\end{coqdoccode}
\ensuremath{\Box} \begin{coqdoccode}
\end{coqdoccode}
\section{自然数列表}





  由任意多个自然数构成的\textit{'自然数列表'}类型
  需要使用递归来定义。
  一个自然数列表有且仅有两种构造方式:

\begin{itemize}
\item  空列表是自然数列表,记为 \coqdocvar{nil};

\item  如果 \coqdocvar{l} 是自然数列表,\coqdocvar{n} 是自然数,
    把 \coqdocvar{n} 添加到 \coqdocvar{l} 的表头,可以构成新的列表,记为 \coqdocvar{cons} \coqdocvar{n} \coqdocvar{l}。

\end{itemize}
\begin{coqdoccode}
\coqdocemptyline
\coqdocnoindent
\coqdockw{Inductive} \coqdocvar{natlist} : \coqdockw{Type} :=\coqdoceol
\coqdocindent{1.00em}
\ensuremath{|} \coqdocvar{nil}\coqdoceol
\coqdocindent{1.00em}
\ensuremath{|} \coqdocvar{cons} (\coqdocvar{n} : \coqdocvar{nat}) (\coqdocvar{l} : \coqdocvar{natlist}).\coqdoceol
\coqdocemptyline
\end{coqdoccode}
例如,\coqdocvar{mylist} 是一个三元素列表。\begin{coqdoccode}
\coqdocemptyline
\coqdocnoindent
\coqdockw{Definition} \coqdocvar{mylist} := \coqdocvar{cons} 1 (\coqdocvar{cons} 2 (\coqdocvar{cons} 3 \coqdocvar{nil})).\coqdoceol
\coqdocemptyline
\end{coqdoccode}
  对于较长的列表,要写很多的 \coqdocvar{cons} 与括号,繁琐易错。
  以下三个 \coqdockw{Notation} 声明允许我们:

\begin{itemize}
\item  使用 :: 中缀操作符代替 \coqdocvar{cons}。注意: :: 是右结合的。

\item  使用 [ ] 代替 \coqdocvar{nil}。

\item  使用单重中括号记法代替多重圆括号记法。

\end{itemize}
\begin{coqdoccode}
\coqdocemptyline
\coqdocnoindent
\coqdockw{Notation} "x :: l" := (\coqdocvar{cons} \coqdocvar{x} \coqdocvar{l})\coqdoceol
\coqdocindent{10.50em}
(\coqdoctac{at} \coqdockw{level} 60, \coqdoctac{right} \coqdockw{associativity}).\coqdoceol
\coqdocnoindent
\coqdockw{Notation} "[ ]" := \coqdocvar{nil}.\coqdoceol
\coqdocnoindent
\coqdockw{Notation} "[ x ; .. ; y ]" := (\coqdocvar{cons} \coqdocvar{x} .. (\coqdocvar{cons} \coqdocvar{y} \coqdocvar{nil}) ..).\coqdoceol
\coqdocemptyline
\coqdocnoindent
\coqdockw{Definition} \coqdocvar{mylist1} := 1 :: (2 :: (3 :: \coqdocvar{nil})).\coqdoceol
\coqdocnoindent
\coqdockw{Definition} \coqdocvar{mylist2} := 1 :: 2 :: 3 :: \coqdocvar{nil}.\coqdoceol
\coqdocnoindent
\coqdockw{Definition} \coqdocvar{mylist3} := [1;2;3].\coqdoceol
\coqdocemptyline
\end{coqdoccode}
  接下来,我们定义一些常用的列表操作函数。
\begin{coqdoccode}
\end{coqdoccode}
\subsubsection{Head(带默认值)与 Tail}





  \coqdocvar{hd} 函数返回列表 \coqdocvar{l} 的第一个元素(即“表头 (head)”)。
  由于空表没有表头,\coqdocvar{hd} 接受另一个参数 \coqdocvar{default} 
  作为这种特殊情况下的默认返回值。
  (后面,我们会学习一种更优雅的处理方式。)


  该函数的定义展示了如何对列表进行模式匹配:

\begin{itemize}
\item  空列表 \coqdocvar{nil};

\item  非空列表 \coqdocvar{l} 可以拆分为表头 \coqdocvar{h}
    与表尾 \coqdocvar{t} (tail; 仍是列表) 两部分。

\end{itemize}
  这种模式匹配很常用。
\begin{coqdoccode}
\coqdocemptyline
\coqdocnoindent
\coqdockw{Definition} \coqdocvar{hd} (\coqdocvar{default} : \coqdocvar{nat}) (\coqdocvar{l} : \coqdocvar{natlist}) : \coqdocvar{nat} :=\coqdoceol
\coqdocindent{1.00em}
\coqdockw{match} \coqdocvar{l} \coqdockw{with}\coqdoceol
\coqdocindent{1.00em}
\ensuremath{|} \coqdocvar{nil} \ensuremath{\Rightarrow} \coqdocvar{default}\coqdoceol
\coqdocindent{1.00em}
\ensuremath{|} \coqdocvar{h} :: \coqdocvar{t} \ensuremath{\Rightarrow} \coqdocvar{h}\coqdoceol
\coqdocindent{1.00em}
\coqdockw{end}.\coqdoceol
\coqdocemptyline
\end{coqdoccode}
\coqdocvar{tl} 函数返回列表 \coqdocvar{l} 除表头以外的部分(即“表尾 (tail)”)。\begin{coqdoccode}
\coqdocnoindent
\coqdockw{Definition} \coqdocvar{tl} (\coqdocvar{l} : \coqdocvar{natlist}) : \coqdocvar{natlist} :=\coqdoceol
\coqdocindent{1.00em}
\coqdockw{match} \coqdocvar{l} \coqdockw{with}\coqdoceol
\coqdocindent{1.00em}
\ensuremath{|} \coqdocvar{nil} \ensuremath{\Rightarrow} \coqdocvar{nil}\coqdoceol
\coqdocindent{1.00em}
\ensuremath{|} \coqdocvar{h} :: \coqdocvar{t} \ensuremath{\Rightarrow} \coqdocvar{t}\coqdoceol
\coqdocindent{1.00em}
\coqdockw{end}.\coqdoceol
\coqdocemptyline
\coqdocnoindent
\coqdockw{Example} \coqdocvar{test\_hd1} : \coqdocvar{hd} 0 [1;2;3] = 1.\coqdoceol
\coqdocnoindent
\coqdockw{Proof}. \coqdoctac{reflexivity}. \coqdockw{Qed}.\coqdoceol
\coqdocnoindent
\coqdockw{Example} \coqdocvar{test\_hd2} : \coqdocvar{hd} 0 [] = 0.\coqdoceol
\coqdocnoindent
\coqdockw{Proof}. \coqdoctac{reflexivity}. \coqdockw{Qed}.\coqdoceol
\coqdocnoindent
\coqdockw{Example} \coqdocvar{test\_tl} : \coqdocvar{tl} [1;2;3] = [2;3].\coqdoceol
\coqdocnoindent
\coqdockw{Proof}. \coqdoctac{reflexivity}. \coqdockw{Qed}.\coqdoceol
\end{coqdoccode}
\subsubsection{Repeat}





  \coqdoctac{repeat} 函数接受自然数 \coqdocvar{n} 和 \coqdocvar{count},
  返回一个包含了 \coqdocvar{count} 个 \coqdocvar{n} 的列表。
\begin{coqdoccode}
\coqdocemptyline
\coqdocnoindent
\coqdockw{Fixpoint} \coqdoctac{repeat} (\coqdocvar{n} \coqdocvar{count} : \coqdocvar{nat}) : \coqdocvar{natlist} :=\coqdoceol
\coqdocindent{1.00em}
\coqdockw{match} \coqdocvar{count} \coqdockw{with}\coqdoceol
\coqdocindent{1.00em}
\ensuremath{|} \coqdocvar{O} \ensuremath{\Rightarrow} \coqdocvar{nil}\coqdoceol
\coqdocindent{1.00em}
\ensuremath{|} \coqdocvar{S} \coqdocvar{count'} \ensuremath{\Rightarrow} \coqdocvar{n} :: (\coqdoctac{repeat} \coqdocvar{n} \coqdocvar{count'})\coqdoceol
\coqdocindent{1.00em}
\coqdockw{end}.\coqdoceol
\end{coqdoccode}
\subsubsection{Length}



 \coqdocvar{length} 函数返回列表 \coqdocvar{l} 的长度。\begin{coqdoccode}
\coqdocemptyline
\coqdocnoindent
\coqdockw{Fixpoint} \coqdocvar{length} (\coqdocvar{l} : \coqdocvar{natlist}) : \coqdocvar{nat} :=\coqdoceol
\coqdocindent{1.00em}
\coqdockw{match} \coqdocvar{l} \coqdockw{with}\coqdoceol
\coqdocindent{1.00em}
\ensuremath{|} \coqdocvar{nil} \ensuremath{\Rightarrow} \coqdocvar{O}\coqdoceol
\coqdocindent{1.00em}
\ensuremath{|} \coqdocvar{h} :: \coqdocvar{t} \ensuremath{\Rightarrow} \coqdocvar{S} (\coqdocvar{length} \coqdocvar{t})\coqdoceol
\coqdocindent{1.00em}
\coqdockw{end}.\coqdoceol
\end{coqdoccode}
\subsubsection{Append}



 \coqdocvar{app} 函数将两个列表 \coqdocvar{l1} \coqdocvar{l2} 联接起来。 \begin{coqdoccode}
\coqdocemptyline
\coqdocnoindent
\coqdockw{Fixpoint} \coqdocvar{app} (\coqdocvar{l1} \coqdocvar{l2} : \coqdocvar{natlist}) : \coqdocvar{natlist} :=\coqdoceol
\coqdocindent{1.00em}
\coqdockw{match} \coqdocvar{l1} \coqdockw{with}\coqdoceol
\coqdocindent{1.00em}
\ensuremath{|} \coqdocvar{nil}    \ensuremath{\Rightarrow} \coqdocvar{l2}\coqdoceol
\coqdocindent{1.00em}
\ensuremath{|} \coqdocvar{h} :: \coqdocvar{t} \ensuremath{\Rightarrow} \coqdocvar{h} :: (\coqdocvar{app} \coqdocvar{t} \coqdocvar{l2})\coqdoceol
\coqdocindent{1.00em}
\coqdockw{end}.\coqdoceol
\coqdocemptyline
\end{coqdoccode}
我们常用右结合的中缀运算符 ++ 代替 \coqdocvar{app}。\begin{coqdoccode}
\coqdocemptyline
\coqdocnoindent
\coqdockw{Notation} "x ++ y" := (\coqdocvar{app} \coqdocvar{x} \coqdocvar{y})\coqdoceol
\coqdocindent{10.50em}
(\coqdoctac{right} \coqdockw{associativity}, \coqdoctac{at} \coqdockw{level} 60).\coqdoceol
\coqdocemptyline
\coqdocnoindent
\coqdockw{Example} \coqdocvar{test\_app1}: [1;2;3] ++ [4;5] = [1;2;3;4;5].\coqdoceol
\coqdocnoindent
\coqdockw{Proof}. \coqdoctac{reflexivity}. \coqdockw{Qed}.\coqdoceol
\coqdocnoindent
\coqdockw{Example} \coqdocvar{test\_app2}: \coqdocvar{nil} ++ [4;5] = [4;5].\coqdoceol
\coqdocnoindent
\coqdockw{Proof}. \coqdoctac{reflexivity}. \coqdockw{Qed}.\coqdoceol
\coqdocnoindent
\coqdockw{Example} \coqdocvar{test\_app3}: [1;2;3] ++ \coqdocvar{nil} = [1;2;3].\coqdoceol
\coqdocnoindent
\coqdockw{Proof}. \coqdoctac{reflexivity}. \coqdockw{Qed}.\coqdoceol
\end{coqdoccode}
\subsubsection{练习}



\paragraph{练习:2 星, standard, recommended (list\_funs)}



  完成函数 \coqdocvar{nonzeros}、\coqdocvar{oddmembers} 和 \coqdocvar{countoddmembers}
  的定义。你可以通过测试用例来理解这些函数的功能。
\begin{coqdoccode}
\coqdocemptyline
\coqdocnoindent
\coqdockw{Fixpoint} \coqdocvar{nonzeros} (\coqdocvar{l} : \coqdocvar{natlist}) : \coqdocvar{natlist}\coqdoceol
\coqdocindent{1.00em}
.\coqdoceol
\coqdocnoindent
\coqdocvar{Admitted}.\coqdoceol
\coqdocemptyline
\coqdocnoindent
\coqdockw{Example} \coqdocvar{test\_nonzeros}:\coqdoceol
\coqdocindent{1.00em}
\coqdocvar{nonzeros} [0;1;0;2;3;0;0] = [1;2;3].\coqdoceol
\coqdocnoindent
\coqdockw{Proof}.\coqdoceol
\coqdocnoindent
\coqdocvar{Admitted}.\coqdoceol
\coqdocemptyline
\coqdocnoindent
\coqdockw{Fixpoint} \coqdocvar{oddmembers} (\coqdocvar{l} : \coqdocvar{natlist}) : \coqdocvar{natlist}\coqdoceol
\coqdocindent{1.00em}
.\coqdoceol
\coqdocnoindent
\coqdocvar{Admitted}.\coqdoceol
\coqdocemptyline
\coqdocnoindent
\coqdockw{Example} \coqdocvar{test\_oddmembers}:\coqdoceol
\coqdocindent{1.00em}
\coqdocvar{oddmembers} [0;1;0;2;3;0;0] = [1;3].\coqdoceol
\coqdocnoindent
\coqdockw{Proof}.\coqdoceol
\coqdocnoindent
\coqdocvar{Admitted}.\coqdoceol
\coqdocemptyline
\coqdocnoindent
\coqdockw{Definition} \coqdocvar{countoddmembers} (\coqdocvar{l} : \coqdocvar{natlist}) : \coqdocvar{nat}\coqdoceol
\coqdocindent{1.00em}
.\coqdoceol
\coqdocnoindent
\coqdocvar{Admitted}.\coqdoceol
\coqdocemptyline
\coqdocnoindent
\coqdockw{Example} \coqdocvar{test\_countoddmembers1}:\coqdoceol
\coqdocindent{1.00em}
\coqdocvar{countoddmembers} [1;0;3;1;4;5] = 4.\coqdoceol
\coqdocnoindent
\coqdockw{Proof}.\coqdoceol
\coqdocnoindent
\coqdocvar{Admitted}.\coqdoceol
\coqdocemptyline
\coqdocnoindent
\coqdockw{Example} \coqdocvar{test\_countoddmembers2}:\coqdoceol
\coqdocindent{1.00em}
\coqdocvar{countoddmembers} [0;2;4] = 0.\coqdoceol
\coqdocnoindent
\coqdockw{Proof}.\coqdoceol
\coqdocnoindent
\coqdocvar{Admitted}.\coqdoceol
\coqdocemptyline
\coqdocnoindent
\coqdockw{Example} \coqdocvar{test\_countoddmembers3}:\coqdoceol
\coqdocindent{1.00em}
\coqdocvar{countoddmembers} \coqdocvar{nil} = 0.\coqdoceol
\coqdocnoindent
\coqdockw{Proof}.\coqdoceol
\coqdocnoindent
\coqdocvar{Admitted}.\coqdoceol
\end{coqdoccode}
\ensuremath{\Box} 

\paragraph{练习:3 星, advanced (alternate)}



  请完成函数 \coqdocvar{alternate} 的定义。
  它交替地从两个列表 \coqdocvar{l1} \coqdocvar{l2} 取元素,
  生成一个合并后的列表。你可以通过测试用例来理解它的功能。
\begin{coqdoccode}
\coqdocemptyline
\coqdocnoindent
\coqdockw{Fixpoint} \coqdocvar{alternate} (\coqdocvar{l1} \coqdocvar{l2} : \coqdocvar{natlist}) : \coqdocvar{natlist}\coqdoceol
\coqdocindent{1.00em}
.\coqdoceol
\coqdocnoindent
\coqdocvar{Admitted}.\coqdoceol
\coqdocemptyline
\coqdocnoindent
\coqdockw{Example} \coqdocvar{test\_alternate1}:\coqdoceol
\coqdocindent{1.00em}
\coqdocvar{alternate} [1;2;3] [4;5;6] = [1;4;2;5;3;6].\coqdoceol
\coqdocnoindent
\coqdocvar{Admitted}.\coqdoceol
\coqdocemptyline
\coqdocnoindent
\coqdockw{Example} \coqdocvar{test\_alternate2}:\coqdoceol
\coqdocindent{1.00em}
\coqdocvar{alternate} [1] [4;5;6] = [1;4;5;6].\coqdoceol
\coqdocnoindent
\coqdocvar{Admitted}.\coqdoceol
\coqdocemptyline
\coqdocnoindent
\coqdockw{Example} \coqdocvar{test\_alternate3}:\coqdoceol
\coqdocindent{1.00em}
\coqdocvar{alternate} [1;2;3] [4] = [1;4;2;3].\coqdoceol
\coqdocnoindent
\coqdocvar{Admitted}.\coqdoceol
\coqdocemptyline
\coqdocnoindent
\coqdockw{Example} \coqdocvar{test\_alternate4}:\coqdoceol
\coqdocindent{1.00em}
\coqdocvar{alternate} [] [20;30] = [20;30].\coqdoceol
\coqdocnoindent
\coqdocvar{Admitted}.\coqdoceol
\end{coqdoccode}
\ensuremath{\Box} \paragraph{练习:3 星, standard, recommended (more list functions)}

 完成函数 \coqdocvar{count} 的定义。\begin{coqdoccode}
\coqdocnoindent
\coqdockw{Fixpoint} \coqdocvar{count} (\coqdocvar{v} : \coqdocvar{nat}) (\coqdocvar{l} : \coqdocvar{natlist}) : \coqdocvar{nat}\coqdoceol
\coqdocindent{1.00em}
.\coqdoceol
\coqdocnoindent
\coqdocvar{Admitted}.\coqdoceol
\coqdocemptyline
\coqdocnoindent
\coqdockw{Example} \coqdocvar{test\_count1}: \coqdocvar{count} 1 [1;2;3;1;4;1] = 3.\coqdoceol
\coqdocnoindent
\coqdocvar{Admitted}.\coqdoceol
\coqdocnoindent
\coqdockw{Example} \coqdocvar{test\_count2}: \coqdocvar{count} 6 [1;2;3;1;4;1] = 0.\coqdoceol
\coqdocnoindent
\coqdocvar{Admitted}.\coqdoceol
\coqdocemptyline
\end{coqdoccode}
完成函数 \coqdocvar{member} 的定义。\begin{coqdoccode}
\coqdocnoindent
\coqdockw{Fixpoint} \coqdocvar{member} (\coqdocvar{v} : \coqdocvar{nat}) (\coqdocvar{l} : \coqdocvar{natlist}) : \coqdocvar{bool}\coqdoceol
\coqdocindent{1.00em}
.\coqdoceol
\coqdocnoindent
\coqdocvar{Admitted}.\coqdoceol
\coqdocemptyline
\coqdocnoindent
\coqdockw{Example} \coqdocvar{test\_member1}: \coqdocvar{member} 1 [1;4;1] = \coqdocvar{true}.\coqdoceol
\coqdocnoindent
\coqdocvar{Admitted}.\coqdoceol
\coqdocemptyline
\coqdocnoindent
\coqdockw{Example} \coqdocvar{test\_member2}: \coqdocvar{member} 2 [1;4;1] = \coqdocvar{false}.\coqdoceol
\coqdocnoindent
\coqdocvar{Admitted}.\coqdoceol
\coqdocemptyline
\end{coqdoccode}
完成函数 \coqdocvar{remov\_one} 的定义。\begin{coqdoccode}
\coqdocnoindent
\coqdockw{Fixpoint} \coqdocvar{remove\_one} (\coqdocvar{v} : \coqdocvar{nat}) (\coqdocvar{l} : \coqdocvar{natlist}) : \coqdocvar{natlist}\coqdoceol
\coqdocindent{1.00em}
.\coqdoceol
\coqdocnoindent
\coqdocvar{Admitted}.\coqdoceol
\coqdocemptyline
\coqdocnoindent
\coqdockw{Example} \coqdocvar{test\_remove\_one1}:\coqdoceol
\coqdocindent{1.00em}
\coqdocvar{count} 5 (\coqdocvar{remove\_one} 5 [2;1;5;4;1]) = 0.\coqdoceol
 \coqdocvar{Admitted}.\coqdoceol
\coqdocemptyline
\coqdocnoindent
\coqdockw{Example} \coqdocvar{test\_remove\_one2}:\coqdoceol
\coqdocindent{1.00em}
\coqdocvar{count} 5 (\coqdocvar{remove\_one} 5 [2;1;4;1]) = 0.\coqdoceol
 \coqdocvar{Admitted}.\coqdoceol
\coqdocemptyline
\coqdocnoindent
\coqdockw{Example} \coqdocvar{test\_remove\_one3}:\coqdoceol
\coqdocindent{1.00em}
\coqdocvar{count} 4 (\coqdocvar{remove\_one} 5 [2;1;4;5;1;4]) = 2.\coqdoceol
 \coqdocvar{Admitted}.\coqdoceol
\coqdocemptyline
\coqdocnoindent
\coqdockw{Example} \coqdocvar{test\_remove\_one4}:\coqdoceol
\coqdocindent{1.00em}
\coqdocvar{count} 5 (\coqdocvar{remove\_one} 5 [2;1;5;4;5;1;4]) = 1.\coqdoceol
 \coqdocvar{Admitted}.\coqdoceol
\coqdocemptyline
\end{coqdoccode}
完成函数 \coqdocvar{remov\_all} 的定义。\begin{coqdoccode}
\coqdocnoindent
\coqdockw{Fixpoint} \coqdocvar{remove\_all} (\coqdocvar{v} : \coqdocvar{nat}) (\coqdocvar{l} : \coqdocvar{natlist}) : \coqdocvar{natlist}\coqdoceol
\coqdocindent{1.00em}
. \coqdocvar{Admitted}.\coqdoceol
\coqdocemptyline
\coqdocnoindent
\coqdockw{Example} \coqdocvar{test\_remove\_all1}:\coqdoceol
\coqdocindent{1.00em}
\coqdocvar{count} 5 (\coqdocvar{remove\_all} 5 [2;1;5;4;1]) = 0.\coqdoceol
\coqdocnoindent
\coqdocvar{Admitted}.\coqdoceol
\coqdocnoindent
\coqdockw{Example} \coqdocvar{test\_remove\_all2}:\coqdoceol
\coqdocindent{1.00em}
\coqdocvar{count} 5 (\coqdocvar{remove\_all} 5 [2;1;4;1]) = 0.\coqdoceol
\coqdocnoindent
\coqdocvar{Admitted}.\coqdoceol
\coqdocnoindent
\coqdockw{Example} \coqdocvar{test\_remove\_all3}:\coqdoceol
\coqdocindent{1.00em}
\coqdocvar{count} 4 (\coqdocvar{remove\_all} 5 [2;1;4;5;1;4]) = 2.\coqdoceol
\coqdocnoindent
\coqdocvar{Admitted}.\coqdoceol
\coqdocnoindent
\coqdockw{Example} \coqdocvar{test\_remove\_all4}:\coqdoceol
\coqdocindent{1.00em}
\coqdocvar{count} 5 (\coqdocvar{remove\_all} 5 [2;1;5;4;5;1;4;5;1;4]) = 0.\coqdoceol
\coqdocnoindent
\coqdocvar{Admitted}.\coqdoceol
\coqdocemptyline
\end{coqdoccode}
完成函数 \coqdocvar{subset} 的定义。\begin{coqdoccode}
\coqdocnoindent
\coqdockw{Fixpoint} \coqdocvar{subset} (\coqdocvar{l1} : \coqdocvar{natlist}) (\coqdocvar{l2} : \coqdocvar{natlist}) : \coqdocvar{bool}\coqdoceol
\coqdocindent{1.00em}
.\coqdoceol
\coqdocnoindent
\coqdocvar{Admitted}.\coqdoceol
\coqdocemptyline
\coqdocnoindent
\coqdockw{Example} \coqdocvar{test\_subset1}: \coqdocvar{subset} [1;2] [2;1;4;1] = \coqdocvar{true}.\coqdoceol
\coqdocnoindent
\coqdocvar{Admitted}.\coqdoceol
\coqdocnoindent
\coqdockw{Example} \coqdocvar{test\_subset2}: \coqdocvar{subset} [1;2;2] [2;1;4;1] = \coqdocvar{false}.\coqdoceol
\coqdocnoindent
\coqdocvar{Admitted}.\coqdoceol
\end{coqdoccode}
\ensuremath{\Box} \begin{coqdoccode}
\end{coqdoccode}
\section{有关列表的论证}



  接下来,我们使用之前学习过的证明策略
  论证与列表相关的定理。


 对于定理 \coqdocvar{nil\_app},\coqdoctac{reflexivity} 已足够。\begin{coqdoccode}
\coqdocemptyline
\coqdocnoindent
\coqdockw{Theorem} \coqdocvar{nil\_app} : \coqdockw{\ensuremath{\forall}} \coqdocvar{l}:\coqdocvar{natlist},\coqdoceol
\coqdocindent{1.00em}
[] ++ \coqdocvar{l} = \coqdocvar{l}.\coqdoceol
\coqdocnoindent
\coqdockw{Proof}. \coqdoctac{reflexivity}. \coqdockw{Qed}.\coqdoceol
\coqdocemptyline
\end{coqdoccode}
定理 \coqdocvar{tl\_length\_pred} 需要分情况讨论。\begin{coqdoccode}
\coqdocemptyline
\coqdocnoindent
\coqdockw{Theorem} \coqdocvar{tl\_length\_pred} : \coqdockw{\ensuremath{\forall}} \coqdocvar{l} : \coqdocvar{natlist},\coqdoceol
\coqdocindent{1.00em}
\coqdocvar{pred} (\coqdocvar{length} \coqdocvar{l}) = \coqdocvar{length} (\coqdocvar{tl} \coqdocvar{l}).\coqdoceol
\coqdocnoindent
\coqdockw{Proof}.\coqdoceol
\coqdocindent{1.00em}
\coqdoctac{intros} \coqdocvar{l}. \coqdoctac{destruct} \coqdocvar{l} \coqdockw{as} [| \coqdocvar{n} \coqdocvar{l'}].\coqdoceol
\coqdocindent{1.00em}
- \coqdoceol
\coqdocindent{2.00em}
\coqdoctac{reflexivity}.\coqdoceol
\coqdocindent{1.00em}
- \coqdoceol
\coqdocindent{2.00em}
\coqdoctac{reflexivity}.\coqdoceol
\coqdocnoindent
\coqdockw{Qed}.\coqdoceol
\end{coqdoccode}
\subsection{对列表进行归纳}





  \coqdocvar{natlist} 是归纳定义的,
  因此,有关列表的很多定理,都可以使用数学归纳法证明。


  假设我们需要证明命题 \coqdocvar{P} 对任意列表 \coqdocvar{l} 都成立。
  我们可以对列表 \coqdocvar{l} 作归纳:

\begin{itemize}
\item  \coqdocvar{l} = []。此时,我们需要证明 \coqdocvar{P} [] 成立。

\item  \coqdocvar{l} = \coqdocvar{n} :: \coqdocvar{l'}。
    此时,我们需要在归纳假设 \coqdocvar{P} \coqdocvar{l'} 成立的条件下,
    证明 \coqdocvar{P} \coqdocvar{l} 成立。

\end{itemize}




  下面使用数学归纳法证明 \coqdocvar{app} 满足结合律。
\begin{coqdoccode}
\coqdocnoindent
\coqdockw{Theorem} \coqdocvar{app\_assoc} : \coqdockw{\ensuremath{\forall}} \coqdocvar{l1} \coqdocvar{l2} \coqdocvar{l3} : \coqdocvar{natlist},\coqdoceol
\coqdocindent{1.00em}
(\coqdocvar{l1} ++ \coqdocvar{l2}) ++ \coqdocvar{l3} = \coqdocvar{l1} ++ (\coqdocvar{l2} ++ \coqdocvar{l3}).\coqdoceol
\coqdocnoindent
\coqdockw{Proof}.\coqdoceol
\coqdocindent{1.00em}
\coqdoctac{intros} \coqdocvar{l1} \coqdocvar{l2} \coqdocvar{l3}.\coqdoceol
\coqdocindent{1.00em}
\coqdoctac{induction} \coqdocvar{l1} \coqdockw{as} [| \coqdocvar{n} \coqdocvar{l1'} \coqdocvar{IHl1'}]. \coqdocindent{1.00em}
- \coqdoceol
\coqdocindent{2.00em}
\coqdoctac{simpl}. \coqdoctac{reflexivity}.\coqdoceol
\coqdocindent{1.00em}
- \coqdoceol
\coqdocindent{2.00em}
\coqdoctac{simpl}. \coqdoctac{rewrite} \ensuremath{\rightarrow} \coqdocvar{IHl1'}. \coqdoctac{reflexivity}.\coqdoceol
\coqdocnoindent
\coqdockw{Qed}.\coqdoceol
\coqdocemptyline
\end{coqdoccode}
  注意: \coqdoctac{induction} \coqdocvar{l1} \coqdockw{as} [ \ensuremath{|} \coqdocvar{n} \coqdocvar{l1'} \coqdocvar{IHl1'}] 的 \coqdockw{as} 从句
  对应于 \coqdocvar{l} 的两个构造函数:

\begin{itemize}
\item   \ensuremath{|}  左边为空。这是因为构造函数 \coqdocvar{nil} 不含参数,
    且在归纳证明中属于基本情形。

\item   \ensuremath{|}  右边有三个参数 \coqdocvar{n} \coqdocvar{l1'} \coqdocvar{IHl1'}。
    前两个参数对应构造函数 \coqdocvar{cons} 的两个参数,
    分别记录了 \coqdocvar{l1} 的表头 \coqdocvar{n} 与表尾 \coqdocvar{l1'}。
    另外,\coqdocvar{IHl1'} 记录了针对 \coqdocvar{l1'} 的归纳假设,
    即 \coqdocvar{IHl1'}: (\coqdocvar{l1'} ++ \coqdocvar{l2}) ++ \coqdocvar{l3} = \coqdocvar{l1'} ++ \coqdocvar{l2} ++ \coqdocvar{l3}。

\end{itemize}


  请确保你真正理解了 \coqdoctac{induction} \coqdocvar{l1} \coqdockw{as} [ \ensuremath{|} \coqdocvar{n} \coqdocvar{l1'} \coqdocvar{IHl1'}] 
  的含义。后面,我们会看到更复杂的例子。
\begin{coqdoccode}
\end{coqdoccode}
\subsubsection{反转列表}



 函数 \coqdocvar{rev} 将列表 \coqdocvar{l} 反转,它的定义使用了 \coqdocvar{app} 函数。\begin{coqdoccode}
\coqdocemptyline
\coqdocnoindent
\coqdockw{Fixpoint} \coqdocvar{rev} (\coqdocvar{l} : \coqdocvar{natlist}) : \coqdocvar{natlist} :=\coqdoceol
\coqdocindent{1.00em}
\coqdockw{match} \coqdocvar{l} \coqdockw{with}\coqdoceol
\coqdocindent{1.00em}
\ensuremath{|} \coqdocvar{nil}    \ensuremath{\Rightarrow} \coqdocvar{nil}\coqdoceol
\coqdocindent{1.00em}
\ensuremath{|} \coqdocvar{h} :: \coqdocvar{t} \ensuremath{\Rightarrow} \coqdocvar{rev} \coqdocvar{t} ++ [\coqdocvar{h}]\coqdoceol
\coqdocindent{1.00em}
\coqdockw{end}.\coqdoceol
\coqdocemptyline
\coqdocnoindent
\coqdockw{Example} \coqdocvar{test\_rev2}: \coqdocvar{rev} \coqdocvar{nil} = \coqdocvar{nil}.\coqdoceol
\coqdocnoindent
\coqdockw{Proof}. \coqdoctac{reflexivity}. \coqdockw{Qed}.\coqdoceol
\coqdocemptyline
\coqdocnoindent
\coqdockw{Example} \coqdocvar{test\_rev1}: \coqdocvar{rev} [1;2;3] = [3;2;1].\coqdoceol
\coqdocnoindent
\coqdockw{Proof}. \coqdoctac{reflexivity}. \coqdockw{Qed}.\coqdoceol
\end{coqdoccode}
\paragraph{练习:3 星, standard, recommended (more list functions)}



  请证明定理 \coqdocvar{app\_length}。
\begin{coqdoccode}
\coqdocnoindent
\coqdockw{Theorem} \coqdocvar{app\_length} : \coqdockw{\ensuremath{\forall}} \coqdocvar{l1} \coqdocvar{l2} : \coqdocvar{natlist},\coqdoceol
\coqdocindent{1.00em}
\coqdocvar{length} (\coqdocvar{l1} ++ \coqdocvar{l2}) = (\coqdocvar{length} \coqdocvar{l1}) + (\coqdocvar{length} \coqdocvar{l2}).\coqdoceol
\coqdocnoindent
\coqdockw{Proof}.\coqdoceol
\end{coqdoccode}
请在此处解答 \begin{coqdoccode}
\coqdocnoindent
\coqdocvar{Admitted}.\coqdoceol
\coqdocemptyline
\end{coqdoccode}
  请证明定理 \coqdocvar{rev\_length}。
  你可能需要使用 \coqdocvar{app\_length} 与 \coqdocvar{plus\_comm}。
\begin{coqdoccode}
\coqdocnoindent
\coqdockw{Theorem} \coqdocvar{rev\_length} : \coqdockw{\ensuremath{\forall}} \coqdocvar{l} : \coqdocvar{natlist},\coqdoceol
\coqdocindent{1.00em}
\coqdocvar{length} (\coqdocvar{rev} \coqdocvar{l}) = \coqdocvar{length} \coqdocvar{l}.\coqdoceol
\coqdocnoindent
\coqdockw{Proof}.\coqdoceol
\end{coqdoccode}
请在此处解答 \begin{coqdoccode}
\coqdocnoindent
\coqdocvar{Admitted}.\coqdoceol
\end{coqdoccode}
\ensuremath{\Box} \begin{coqdoccode}
\end{coqdoccode}
\subsection{列表练习,第一部分}





  你需要通过大量的练习与思考 (练习之后的思考很重要!很重要!很重要!)
  培养证明的直觉。
  比如,分情形分析够不够用? 需不需要用数学归纳法? 对什么作归纳? 等等。
\paragraph{练习:3 星, standard (list\_exercises)}

\begin{coqdoccode}
\coqdocnoindent
\coqdockw{Theorem} \coqdocvar{app\_nil\_r} : \coqdockw{\ensuremath{\forall}} \coqdocvar{l} : \coqdocvar{natlist},\coqdoceol
\coqdocindent{1.00em}
\coqdocvar{l} ++ [] = \coqdocvar{l}.\coqdoceol
\coqdocnoindent
\coqdockw{Proof}.\coqdoceol
\coqdocnoindent
\coqdocvar{Admitted}.\coqdoceol
\coqdocemptyline
\coqdocnoindent
\coqdockw{Theorem} \coqdocvar{rev\_app\_distr}: \coqdockw{\ensuremath{\forall}} \coqdocvar{l1} \coqdocvar{l2} : \coqdocvar{natlist},\coqdoceol
\coqdocindent{1.00em}
\coqdocvar{rev} (\coqdocvar{l1} ++ \coqdocvar{l2}) = \coqdocvar{rev} \coqdocvar{l2} ++ \coqdocvar{rev} \coqdocvar{l1}.\coqdoceol
\coqdocnoindent
\coqdockw{Proof}.\coqdoceol
\coqdocnoindent
\coqdocvar{Admitted}.\coqdoceol
\coqdocemptyline
\coqdocnoindent
\coqdockw{Theorem} \coqdocvar{rev\_involutive} : \coqdockw{\ensuremath{\forall}} \coqdocvar{l} : \coqdocvar{natlist},\coqdoceol
\coqdocindent{1.00em}
\coqdocvar{rev} (\coqdocvar{rev} \coqdocvar{l}) = \coqdocvar{l}.\coqdoceol
\coqdocnoindent
\coqdockw{Proof}.\coqdoceol
\coqdocnoindent
\coqdocvar{Admitted}.\coqdoceol
\coqdocemptyline
\end{coqdoccode}
\coqdocvar{app\_assoc4} 有简洁的证明。不要走了弯路。\begin{coqdoccode}
\coqdocemptyline
\coqdocnoindent
\coqdockw{Theorem} \coqdocvar{app\_assoc4} : \coqdockw{\ensuremath{\forall}} \coqdocvar{l1} \coqdocvar{l2} \coqdocvar{l3} \coqdocvar{l4} : \coqdocvar{natlist},\coqdoceol
\coqdocindent{1.00em}
\coqdocvar{l1} ++ (\coqdocvar{l2} ++ (\coqdocvar{l3} ++ \coqdocvar{l4})) = ((\coqdocvar{l1} ++ \coqdocvar{l2}) ++ \coqdocvar{l3}) ++ \coqdocvar{l4}.\coqdoceol
\coqdocnoindent
\coqdockw{Proof}.\coqdoceol
\coqdocnoindent
\coqdocvar{Admitted}.\coqdoceol
\coqdocemptyline
\coqdocnoindent
\coqdockw{Print} \coqdocvar{nonzeros}. \coqdocnoindent
\coqdockw{Lemma} \coqdocvar{nonzeros\_app} : \coqdockw{\ensuremath{\forall}} \coqdocvar{l1} \coqdocvar{l2} : \coqdocvar{natlist},\coqdoceol
\coqdocindent{1.00em}
\coqdocvar{nonzeros} (\coqdocvar{l1} ++ \coqdocvar{l2}) = (\coqdocvar{nonzeros} \coqdocvar{l1}) ++ (\coqdocvar{nonzeros} \coqdocvar{l2}).\coqdoceol
\coqdocnoindent
\coqdockw{Proof}.\coqdoceol
\coqdocnoindent
\coqdocvar{Admitted}.\coqdoceol
\end{coqdoccode}
\ensuremath{\Box} 

\paragraph{练习:2 星, standard (eqblist)}



  请完成 \coqdocvar{eqblist} 的定义,它判断列表 \coqdocvar{l1}、\coqdocvar{l2} 是否相同。
\begin{coqdoccode}
\coqdocemptyline
\coqdocnoindent
\coqdockw{Fixpoint} \coqdocvar{eqblist} (\coqdocvar{l1} \coqdocvar{l2} : \coqdocvar{natlist}) : \coqdocvar{bool}\coqdoceol
\coqdocindent{1.00em}
.\coqdoceol
\coqdocnoindent
\coqdocvar{Admitted}.\coqdoceol
\coqdocemptyline
\coqdocnoindent
\coqdockw{Example} \coqdocvar{test\_eqblist1} :\coqdoceol
\coqdocindent{1.00em}
(\coqdocvar{eqblist} \coqdocvar{nil} \coqdocvar{nil} = \coqdocvar{true}).\coqdoceol
\coqdocnoindent
\coqdocvar{Admitted}.\coqdoceol
\coqdocemptyline
\coqdocnoindent
\coqdockw{Example} \coqdocvar{test\_eqblist2} :\coqdoceol
\coqdocindent{1.00em}
\coqdocvar{eqblist} [1;2;3] [1;2;3] = \coqdocvar{true}.\coqdoceol
\coqdocnoindent
\coqdocvar{Admitted}.\coqdoceol
\coqdocemptyline
\coqdocnoindent
\coqdockw{Example} \coqdocvar{test\_eqblist3} :\coqdoceol
\coqdocindent{1.00em}
\coqdocvar{eqblist} [1;2;3] [1;2;4] = \coqdocvar{false}.\coqdoceol
\coqdocnoindent
\coqdocvar{Admitted}.\coqdoceol
\coqdocemptyline
\end{coqdoccode}
  如果我们将函数 \coqdocvar{eqblist} 看作两个列表之间的 \_关系 (Relation)\_,
  那么它是 \_自反的 (Reflexive)\_。


  嗯,如果你现在还不明白上面那句话在说些什么,
  不要紧,直接证明下面的定理 \coqdocvar{eqblist\_refl} 就好了。 
\begin{coqdoccode}
\coqdocnoindent
\coqdockw{Theorem} \coqdocvar{eqblist\_refl} : \coqdockw{\ensuremath{\forall}} \coqdocvar{l} : \coqdocvar{natlist},\coqdoceol
\coqdocindent{1.00em}
\coqdocvar{true} = \coqdocvar{eqblist} \coqdocvar{l} \coqdocvar{l}.\coqdoceol
\coqdocnoindent
\coqdockw{Proof}.\coqdoceol
\coqdocnoindent
\coqdocvar{Admitted}.\coqdoceol
\coqdocemptyline
\end{coqdoccode}
\paragraph{练习:1 星, standard (count\_member\_nonzero)}

\begin{coqdoccode}
\coqdocnoindent
\coqdockw{Theorem} \coqdocvar{count\_member\_nonzero} : \coqdockw{\ensuremath{\forall}} (\coqdocvar{l} : \coqdocvar{natlist}),\coqdoceol
\coqdocindent{1.00em}
1 <=? (\coqdocvar{count} 1 (1 :: \coqdocvar{l})) = \coqdocvar{true}.\coqdoceol
\coqdocnoindent
\coqdockw{Proof}.\coqdoceol
\coqdocnoindent
\coqdocvar{Admitted}.\coqdoceol
\coqdocemptyline
\coqdocnoindent
\coqdockw{Print} \coqdocvar{remove\_one}. \end{coqdoccode}
\paragraph{练习:3 星, advanced (remove\_does\_not\_increase\_count)}

\begin{coqdoccode}
\coqdocnoindent
\coqdockw{Theorem} \coqdocvar{remove\_does\_not\_increase\_count}: \coqdockw{\ensuremath{\forall}} (\coqdocvar{l} : \coqdocvar{natlist}),\coqdoceol
\coqdocindent{1.00em}
(\coqdocvar{count} 0 (\coqdocvar{remove\_one} 0 \coqdocvar{l})) <=? (\coqdocvar{count} 0 \coqdocvar{l}) = \coqdocvar{true}.\coqdoceol
\coqdocnoindent
\coqdockw{Proof}.\coqdoceol
\coqdocnoindent
\coqdocvar{Admitted}.\coqdoceol
\end{coqdoccode}
\ensuremath{\Box} 

\paragraph{练习:4 星, advanced (rev\_injective)}



  请用尽可能简洁的方法证明定理 \coqdocvar{rev\_injective}: \coqdocvar{rev} 是单射函数。
\begin{coqdoccode}
\coqdocnoindent
\coqdockw{Theorem} \coqdocvar{rev\_injective} : \coqdockw{\ensuremath{\forall}} (\coqdocvar{l1} \coqdocvar{l2} : \coqdocvar{natlist}), \coqdoceol
\coqdocindent{1.00em}
\coqdocvar{rev} \coqdocvar{l1} = \coqdocvar{rev} \coqdocvar{l2} \ensuremath{\rightarrow} \coqdocvar{l1} = \coqdocvar{l2}.\coqdoceol
\coqdocnoindent
\coqdockw{Proof}.\coqdoceol
\coqdocnoindent
\coqdocvar{Admitted}.\coqdoceol
\end{coqdoccode}
\ensuremath{\Box} \begin{coqdoccode}
\end{coqdoccode}
\section{Options 可选类型}

\begin{coqdoccode}
\coqdocemptyline
\coqdocnoindent
\coqdockw{Print} \coqdocvar{hd}.\coqdoceol
\end{coqdoccode}
  在本节的最后,我们回到一开始对 \coqdocvar{hd} (head) 函数的定义:
  \coqdockw{Definition} \coqdocvar{hd} (\coqdocvar{default} : \coqdocvar{nat}) (\coqdocvar{l} : \coqdocvar{natlist}) : \coqdocvar{nat}。
  为了处理 \coqdocvar{l} 为空的情况,\coqdocvar{hd} 要求调用者提供默认返回值 \coqdocvar{default} : \coqdocvar{nat}。
  然而,这种处理方式不够优雅:

\begin{itemize}
\item  破坏了 \coqdocvar{hd} 的语义。

\item  返回值为 \coqdocvar{default} 时,无法区分 \coqdocvar{l} 的表头确实为 \coqdocvar{default} 
    与 \coqdocvar{l} 为空的情况。

\item  给调用者增加负担。

\end{itemize}


  函数 \coqdocvar{nth}-\coqdocvar{bad} 是对 \coqdocvar{hd} 的扩展,它返回列表 \coqdocvar{l} 中的第 \coqdocvar{n} 个元素。
  当 \coqdocvar{l} 过短时,它返回一个任意值,这里选择返回 42。
  它存在与 \coqdocvar{hd} 类似的不足。
\begin{coqdoccode}
\coqdocemptyline
\coqdocnoindent
\coqdockw{Fixpoint} \coqdocvar{nth\_bad} (\coqdocvar{l} : \coqdocvar{natlist}) (\coqdocvar{n} : \coqdocvar{nat}) : \coqdocvar{nat} :=\coqdoceol
\coqdocindent{1.00em}
\coqdockw{match} \coqdocvar{l} \coqdockw{with}\coqdoceol
\coqdocindent{1.00em}
\ensuremath{|} \coqdocvar{nil} \ensuremath{\Rightarrow} 42  \coqdoceol
\coqdocindent{1.00em}
\ensuremath{|} \coqdocvar{a} :: \coqdocvar{l'} \ensuremath{\Rightarrow} \coqdockw{match} \coqdocvar{n} =? \coqdocvar{O} \coqdockw{with}\coqdoceol
\coqdocindent{7.50em}
\ensuremath{|} \coqdocvar{true} \ensuremath{\Rightarrow} \coqdocvar{a}\coqdoceol
\coqdocindent{7.50em}
\ensuremath{|} \coqdocvar{false} \ensuremath{\Rightarrow} \coqdocvar{nth\_bad} \coqdocvar{l'} (\coqdocvar{pred} \coqdocvar{n})\coqdoceol
\coqdocindent{7.50em}
\coqdockw{end}\coqdoceol
\coqdocindent{1.00em}
\coqdockw{end}.\coqdoceol
\coqdocemptyline
\coqdocnoindent
\coqdockw{Print} \coqdocvar{option}.\coqdoceol
\end{coqdoccode}
  为了解决该类问题,Coq 提供了 \coqdocvar{option} 类型。
  \coqdocvar{option} 类型是对 \_可选值 (Optional Value)\_ 的一种封装。
  作为函数的返回值类型,它表示该函数可能会返回一个无意义的值,
  用以标识错误处理。
  它包含两个构造函数:

\begin{itemize}
\item  Some A: 表示值 A。

\item  None: 表示空值。

\end{itemize}


  很多程序设计语言里都有类似的 \coqdocvar{option} 类型,
  如 Java 8 中的 \coqdocvar{Optional},Scala 中的 \coqdocvar{Option},
  Haskell 中的 \coqdocvar{Maybe} 等。


  Coq 中的 \coqdocvar{option} 是 \_多态类型 (Polymorphic Type)\_
  (下一节会介绍这个概念)。
  本节我们将被封装的值 A 的类型限定为 \coqdocvar{nat}。
\begin{coqdoccode}
\coqdocemptyline
\coqdocnoindent
\coqdockw{Inductive} \coqdocvar{natoption} : \coqdockw{Type} :=\coqdoceol
\coqdocindent{1.00em}
\ensuremath{|} \coqdocvar{Some} (\coqdocvar{n} : \coqdocvar{nat})\coqdoceol
\coqdocindent{1.00em}
\ensuremath{|} \coqdocvar{None}.\coqdoceol
\coqdocemptyline
\end{coqdoccode}
  \coqdocvar{nth\_error} 是对 \coqdocvar{nth\_bad} 的改进。
  注意,\coqdocvar{nth\_error} 的返回类型是 \coqdocvar{natoption}:

\begin{itemize}
\item  当列表 \coqdocvar{l} 过短时,它返回 \coqdocvar{None},

\item  否则它将元素 \coqdocvar{a} 封装成类型为 \coqdocvar{natoption} 的 \coqdocvar{Some} \coqdocvar{a},
    然后返回 \coqdocvar{Some} \coqdocvar{a}。

\end{itemize}
\begin{coqdoccode}
\coqdocemptyline
\coqdocnoindent
\coqdockw{Fixpoint} \coqdocvar{nth\_error} (\coqdocvar{l} : \coqdocvar{natlist}) (\coqdocvar{n} : \coqdocvar{nat}) : \coqdocvar{natoption} :=\coqdoceol
\coqdocindent{1.00em}
\coqdockw{match} \coqdocvar{l} \coqdockw{with}\coqdoceol
\coqdocindent{1.00em}
\ensuremath{|} \coqdocvar{nil} \ensuremath{\Rightarrow} \coqdocvar{None} \coqdoceol
\coqdocindent{1.00em}
\ensuremath{|} \coqdocvar{a} :: \coqdocvar{l'} \ensuremath{\Rightarrow} \coqdockw{match} \coqdocvar{n} =? \coqdocvar{O} \coqdockw{with}\coqdoceol
\coqdocindent{7.50em}
\ensuremath{|} \coqdocvar{true} \ensuremath{\Rightarrow} \coqdocvar{Some} \coqdocvar{a} \coqdoceol
\coqdocindent{7.50em}
\ensuremath{|} \coqdocvar{false} \ensuremath{\Rightarrow} \coqdocvar{nth\_error} \coqdocvar{l'} (\coqdocvar{pred} \coqdocvar{n})\coqdoceol
\coqdocindent{7.50em}
\coqdockw{end}\coqdoceol
\coqdocindent{1.00em}
\coqdockw{end}.\coqdoceol
\coqdocemptyline
\coqdocnoindent
\coqdockw{Example} \coqdocvar{test\_nth\_error1} : \coqdocvar{nth\_error} [4;5;6;7] 0 = \coqdocvar{Some} 4.\coqdoceol
\coqdocnoindent
\coqdockw{Proof}. \coqdoctac{reflexivity}. \coqdockw{Qed}.\coqdoceol
\coqdocnoindent
\coqdockw{Example} \coqdocvar{test\_nth\_error2} : \coqdocvar{nth\_error} [4;5;6;7] 3 = \coqdocvar{Some} 7.\coqdoceol
\coqdocnoindent
\coqdockw{Proof}. \coqdoctac{reflexivity}. \coqdockw{Qed}.\coqdoceol
\coqdocnoindent
\coqdockw{Example} \coqdocvar{test\_nth\_error3} : \coqdocvar{nth\_error} [4;5;6;7] 9 = \coqdocvar{None}.\coqdoceol
\coqdocnoindent
\coqdockw{Proof}. \coqdoctac{reflexivity}. \coqdockw{Qed}.\coqdoceol
\coqdocemptyline
\end{coqdoccode}
  \coqdocvar{nth\_error} 中的嵌套模式匹配 \coqdockw{match} \coqdocvar{n}=? \coqdocvar{O}
  也可以换成条件表达式,
  如下面的 \coqdocvar{nth\_error\_if} 所示。
\begin{coqdoccode}
\coqdocemptyline
\coqdocnoindent
\coqdockw{Fixpoint} \coqdocvar{nth\_error\_if} (\coqdocvar{l} : \coqdocvar{natlist}) (\coqdocvar{n} : \coqdocvar{nat}) : \coqdocvar{natoption} :=\coqdoceol
\coqdocindent{1.00em}
\coqdockw{match} \coqdocvar{l} \coqdockw{with}\coqdoceol
\coqdocindent{1.00em}
\ensuremath{|} \coqdocvar{nil} \ensuremath{\Rightarrow} \coqdocvar{None}\coqdoceol
\coqdocindent{1.00em}
\ensuremath{|} \coqdocvar{a} :: \coqdocvar{l'} \ensuremath{\Rightarrow} \coqdoceol
\coqdocindent{3.00em}
\coqdockw{if} \coqdocvar{n} =? \coqdocvar{O} \coqdockw{then} \coqdocvar{Some} \coqdocvar{a} \coqdoceol
\coqdocindent{8.00em}
\coqdockw{else} \coqdocvar{nth\_error\_if} \coqdocvar{l'} (\coqdocvar{pred} \coqdocvar{n})\coqdoceol
\coqdocindent{1.00em}
\coqdockw{end}.\coqdoceol
\coqdocemptyline
\end{coqdoccode}
  接收到类型为 \coqdocvar{natoption} 的值 \coqdocvar{v} 以后,
  我们通常会对其进行模式匹配 \coqdockw{match} \coqdocvar{v} \coqdockw{with}:

\begin{itemize}
\item  如果为 \coqdocvar{None},则做特殊处理。

\item  如果为 \coqdocvar{Some} \coqdocvar{a},则对 \coqdocvar{a} 做处理。

\end{itemize}


\paragraph{练习:2 星, standard (hd\_error)}

 请使用 \coqdocvar{natoption} 思想修改之前定义的 \coqdocvar{hd} 函数。\begin{coqdoccode}
\coqdocemptyline
\coqdocnoindent
\coqdockw{Definition} \coqdocvar{hd\_error} (\coqdocvar{l} : \coqdocvar{natlist}) : \coqdocvar{natoption}\coqdoceol
\coqdocindent{1.00em}
.\coqdoceol
\coqdocnoindent
\coqdocvar{Admitted}.\coqdoceol
\coqdocemptyline
\coqdocnoindent
\coqdockw{Example} \coqdocvar{test\_hd\_error1} : \coqdocvar{hd\_error} [] = \coqdocvar{None}.\coqdoceol
 \coqdocvar{Admitted}.\coqdoceol
\coqdocemptyline
\coqdocnoindent
\coqdockw{Example} \coqdocvar{test\_hd\_error2} : \coqdocvar{hd\_error} [1] = \coqdocvar{Some} 1.\coqdoceol
 \coqdocvar{Admitted}.\coqdoceol
\coqdocemptyline
\coqdocnoindent
\coqdockw{Example} \coqdocvar{test\_hd\_error3} : \coqdocvar{hd\_error} [5;6] = \coqdocvar{Some} 5.\coqdoceol
 \coqdocvar{Admitted}.\coqdoceol
\end{coqdoccode}
\ensuremath{\Box} \begin{coqdoccode}
\coqdocnoindent
\coqdockw{End} \coqdocvar{NatList}.\coqdoceol
\end{coqdoccode}
\end{document}
